\documentclass[12pt,a4paper]{report}

% Essential packages
\usepackage[utf8]{inputenc}
\usepackage[T1]{fontenc}
\usepackage{lmodern}
\usepackage{amsmath,amssymb,amsfonts}
\usepackage{graphicx}
\usepackage{hyperref}
\usepackage{xcolor}
\usepackage{booktabs}
\usepackage{listings}
\usepackage{mdframed}
\usepackage{tcolorbox}
\usepackage{enumitem}
\usepackage{tikz}
\usetikzlibrary{shapes,arrows,positioning}

% Custom settings
\hypersetup{
    colorlinks=true,
    linkcolor=blue,
    filecolor=magenta,
    urlcolor=cyan,
}

% Code listing settings
\lstset{
  backgroundcolor=\color{gray!10},
  basicstyle=\ttfamily\footnotesize,
  breakatwhitespace=false,
  breaklines=true,
  captionpos=b,
  commentstyle=\color{green!60!black},
  keywordstyle=\color{blue},
  stringstyle=\color{purple},
  numbers=left,
  numberstyle=\tiny\color{gray},
  stepnumber=1,
  tabsize=2,
  title=\lstname,
  frame=single,
  framesep=5pt,
  framexleftmargin=5pt,
  framexrightmargin=5pt,
  xleftmargin=15pt,
}

% Definition environment
\newmdenv[
  linecolor=blue!70,
  backgroundcolor=blue!5,
  roundcorner=5pt,
  skipabove=\baselineskip,
  skipbelow=\baselineskip,
  innertopmargin=\baselineskip,
  innerbottommargin=\baselineskip
]{definition}

% Key concept environment
\newtcolorbox{keyconcept}{
  colback=green!5,
  colframe=green!70!black,
  fonttitle=\bfseries,
  title=Key Concept,
  rounded corners
}

% Important note environment
\newtcolorbox{important}{
  colback=red!5,
  colframe=red!70!black,
  fonttitle=\bfseries,
  title=Important Note,
  rounded corners
}

\title{
  \Huge{\textbf{Real-Time and Big Data Analytics}} \\
  \vspace{0.5cm}
  \Large{Course Notes} \\
  \vspace{0.5cm}
  \large{Spring 2025}
}
\author{Your Name}
\date{\today}

\begin{document}

\maketitle
\tableofcontents

\chapter{Introduction to Real-Time and Big Data Analytics}
% Content for Chapter 1 would go here

\chapter{Big Data Storage}
\section{Definition of Big Data}
Big Data is when \textbf{the size of the data itself} becomes part of the problem.

\begin{keyconcept}
One 10 TB hard disk drive (HDD) is less effective than 100 HDDs distributed across 20 computers due to \textbf{CPU limitations} when dealing with big data processing.
\end{keyconcept}

\section{Big Data Storage Concepts}
Big Data analytics uses \textbf{highly scalable distributed technologies and frameworks}. To store Big Data datasets, \textbf{often in multiple copies}, innovative storage strategies and technologies have been created to achieve \textbf{cost-effective} and \textbf{highly scalable} storage solutions.

Key concepts we'll explore include:
\begin{itemize}
  \item Clusters
  \item Distributed file systems
  \item Relational database management systems (RDBMS)
  \item NoSQL databases
  \item Sharding
  \item Replication
  \item CAP theorem
  \item ACID
  \item BASE
\end{itemize}

\section{Clusters}
A cluster is a tightly coupled \textbf{collection of servers (``nodes'')}. These servers usually have similar hardware specifications and are \textbf{connected together via a network} to work as a single unit.

Each node in the cluster has its \textbf{own dedicated resources}, such as CPU, memory, and storage.

A cluster can \textbf{execute a job} by \textbf{splitting it into small tasks} and \textbf{distributing their execution onto different computers} that belong to the cluster.

\begin{mdframed}
Cluster → Racks → Nodes (servers) → CPU, Memory, Storage
\end{mdframed}

\section{Distributed File Systems}
A file is \textbf{the most basic unit of storage} to store data.

A file system (FS) is the method of \textbf{organizing files} on a storage device.

A distributed file system (DFS) is a file system that can \textbf{store large files spread across multiple nodes of a cluster}.

\begin{mdframed}
Examples: Google File System (GFS), Hadoop Distributed File System (HDFS), Amazon S3, and Azure Blob Storage.
\end{mdframed}

\section{Database Systems for Big Data}

\subsection{Relational Database Management Systems (RDBMS)}
A relational database management system (RDBMS) is a product that presents a view of data as \textbf{a collection of rows and columns}.

SQL (structured query language) is the standard language for \textbf{querying} and \textbf{maintaining} the database.

A transaction symbolizes \textbf{a unit of work} that is performed against a database, and treated in a \textbf{coherent and reliable way} independent of other transactions.

\subsection{NoSQL Databases}
A NoSQL database is a \textbf{non-relational database} that provides a mechanism for \textbf{storage and retrieval of data} that is \textbf{highly scalable, fault-tolerant} and specifically designed to house \textbf{semi-structured} and \textbf{unstructured} data.

\begin{mdframed}
Examples of NoSQL database types:
\begin{itemize}
  \item Key-Value stores (e.g., Redis)
  \item Document stores (e.g., MongoDB)
  \item Wide-Column stores (e.g., Cassandra)
  \item Graph databases (e.g., Pregel)
\end{itemize}
\end{mdframed}

\section{Distributed Data Management Strategies}

\subsection{Sharding}
Sharding is the process of \textbf{horizontally partitioning} a large dataset into a collection of smaller, more manageable pieces called \textbf{shards}.

Each shard is stored on a \textbf{separate node}, and is responsible for \textbf{only the data stored on that node}.

All shards share \textbf{the same schema}, but each shard contains a \textbf{subset of the data}.

\subsubsection{How Sharding Works in Practice}
\begin{enumerate}
  \item Each shard can \textbf{independently} service reads and writes for the \textbf{specific subset} of data that it is responsible for.
  \item Depending on the query, data may need to be fetched from \textbf{multiple shards}.
\end{enumerate}

\begin{keyconcept}
\textbf{Benefits of Sharding:}
\begin{itemize}
  \item Horizontal scalability: Sharding allows for \textbf{scaling out} by adding more nodes to the cluster, which can help to \textbf{distribute the load} and improve performance.
  \item Sharding provides \textbf{partial tolerance towards failure}.
\end{itemize}
\end{keyconcept}

\begin{important}
\textbf{Concerns of Sharding:}
\begin{itemize}
  \item Queries requiring data from multiple shards can be \textbf{slower} and will impose performance penalties.
  \item To mitigate such performance penalties, \textbf{data locality} keeps commonly accessed data in the same shard.
\end{itemize}
\end{important}

\subsection{Replication}
Replication stores multiple copies of a dataset on multiple nodes.

\subsubsection{Methods of Replication}
\begin{itemize}
  \item Master-slave replication
  \item Peer-to-peer replication
\end{itemize}

\subsubsection{Master-Slave Replication}
All data is written to a \textbf{master node}. Once saved, the data is replicated over to multiple \textbf{slave nodes}.
\begin{itemize}
  \item \textbf{Write requests}, including insert, update, and delete, occur on the \textbf{master node}.
  \item \textbf{Read requests} can occur on either the \textbf{master node} or any of the \textbf{slave nodes}.
\end{itemize}

Master-slave replication is ideal for \textbf{read-intensive loads}. Growing read demands can be managed by \textbf{horizontal scaling} to add more slave nodes.

Writes are \textbf{consistent}:
\begin{itemize}
  \item All writes are coordinated by the \textbf{master node}.
  \item However, \textbf{write performance will suffer} as the amount of writes increases.
\end{itemize}

If the \textbf{master node fails}:
\begin{itemize}
  \item Reads are still possible from the \textbf{slave nodes}.
  \item Writes are \textbf{not possible} until a new master node is reestablished.
\end{itemize}

Recovery options:
\begin{itemize}
  \item Resurrect the master node from a \textbf{backup}.
  \item Choose a new master node from the \textbf{slave nodes}.
\end{itemize}

\subsubsection{Peer-to-Peer Replication}
All nodes operate at the \textbf{same level}. Each peer is \textbf{equally capable} of handling read and write requests. Each \textbf{write} is copied to \textbf{all peers}.

\begin{important}
Concern: \textbf{Read/write inconsistency}
\end{important}

Strategies to address inconsistency:
\begin{itemize}
  \item \textbf{Pessimistic concurrency} is a \textbf{proactive} strategy, using \textbf{locking} to ensure that only one update to a record can occur at a time. However, this is \textbf{detrimental to availability} since the database record being updated remains unavailable to other users until the lock is released.
  \item \textbf{Optimistic concurrency} is a \textbf{reactive} strategy that \textbf{does not use locking}. Instead, it \textbf{allows inconsistency} to occur with knowledge that \textbf{eventually} consistency will be achieved after all updates have propagated.
\end{itemize}

\subsection{Combined Approaches for Data Distribution}
\begin{itemize}
  \item \textbf{Sharding and master-slave replication:} Each node acts both as a \textbf{master} and a \textbf{slave} for \textbf{different shards}.
  \item \textbf{Sharding and peer-to-peer replication:} Each node contains \textbf{replicas} of two different \textbf{shards}.
\end{itemize}

\section{Theoretical Foundations and Design Principles}

\subsection{CAP Theorem}
A distributed system may \textbf{wish} to provide three guarantees:
\begin{itemize}
  \item \textbf{C}onsistency: A read from any node results in the \textbf{same, most recently written} data across multiple nodes.
  \item \textbf{A}vailability: A read/write request will \textbf{always be acknowledged} in the form of a success or failure, regardless of the state of the system.
  \item \textbf{P}artition tolerance: The database system can \textbf{tolerate communication outages} that split the cluster into multiple silos and can still \textbf{service read/write requests}.
\end{itemize}

\begin{keyconcept}
The CAP theorem states that a distributed system can only provide \textbf{two} of the three guarantees at any given time.
\end{keyconcept}

\subsection{Database Design Principles}

\subsubsection{ACID}
ACID is a \textbf{traditional} database design principle for \textbf{transaction management}.

\begin{itemize}
  \item \textbf{A}tomicity: Ensures that all transactions will always succeed or fail \textbf{completely}.
  \item \textbf{C}onsistency: Ensures that only data that \textbf{conforms to the constraints of the database schema} can be written to the database.
  \item \textbf{I}solation: Ensures that the results of a transaction are \textbf{not visible} to other transactions until the transaction is committed.
  \item \textbf{D}urability: Ensures the results of a transaction are \textbf{permanent}, regardless of any system failures.
\end{itemize}

Traditional databases leverage pessimistic concurrency controls (i.e., locking) to ensure ACID compliance. Database systems providing traditional ACID guarantees choose \textbf{consistency} over \textbf{availability}.

\subsubsection{BASE}
BASE is a database design principle leveraged by many \textbf{distributed} systems.

\begin{itemize}
  \item \textbf{B}asically \textbf{A}vailable: The system is \textbf{always available} to service read/write requests, even if the data is not consistent.
  \item \textbf{S}oft state: The system may be in a \textbf{temporary inconsistent state} at any given time, but will eventually become consistent.
  \item \textbf{E}ventual consistency: The system will \textbf{eventually become consistent} after all updates have propagated.
\end{itemize}

When a database supports BASE, it favors \textbf{availability} over \textbf{consistency}. BASE leverages optimistic concurrency by relaxing the strong consistency constraints mandated by the ACID properties.

\subsubsection{ACID vs BASE Comparison}
ACID ensures \textbf{immediate consistency at the expense of availability} due to record locking. 

BASE emphasizes \textbf{availability over immediate consistency}.

\chapter{Distributed Computing Models}
% Content for Chapter 3 would go here

\chapter{Real-Time Analytics}
% Content for Chapter 4 would go here

\chapter{Big Data Visualization}
% Content for Chapter 5 would go here

\chapter{Case Studies and Applications}
% Content for Chapter 6 would go here

\appendix
\chapter{Glossary of Terms}
% Glossary would go here

\chapter{Important Formulas and Theorems}
% Formulas and theorems would go here

\end{document}